\usepackage[T1]{fontenc}
\usepackage{setspace}
\usepackage{lastpage}
\setlength{\parskip}{.5em}

\usepackage{lmodern}
\usepackage{amssymb,amsmath}
\usepackage{helvet}
\renewcommand{\familydefault}{\sfdefault}


\usepackage[tablegrid,owncaptions]{vhistory}
\renewcommand \vhAuthorColWidth{0.9\hsize}
\renewcommand \vhChangeColWidth{1.1\hsize}
\renewcommand{\vhhistoryname}{Change history}
\renewcommand{\vhversionname}{Issue}

% \usepackage[headsepline=0.4pt,footsepline=0.4pt]{scrlayer-scrpage}
\usepackage{scrlayer-scrpage}

\usepackage{mdframed}
\usepackage{graphicx}
\usepackage{wrapfig}
% \graphicspath{{./graphics/}{./png/}}

\usepackage[paper=a4paper,left=25mm,right=25mm,top=40mm,foot=12.5mm]{geometry}

\usepackage{tabularx}
\usepackage{multirow}
\usepackage[toc,page]{appendix}
\usepackage{enumerate}
\usepackage{cite}
\usepackage{lipsum}
\usepackage{float}
\usepackage{bbding}
\usepackage[rgb,dvipsnames,table]{xcolor}

% define code style
\usepackage{listings}
\usepackage{color}
\definecolor{mygreen}{rgb}{0,0.6,0}
\definecolor{mygray}{rgb}{0.5,0.5,0.5}
\definecolor{mymauve}{rgb}{0.58,0,0.82}

\lstset{ 
  backgroundcolor=\color{white},   % choose the background color; you must add \usepackage{color} or \usepackage{xcolor}; should come as last argument
  basicstyle=\footnotesize,        % the size of the fonts that are used for the code
  breakatwhitespace=false,         % sets if automatic breaks should only happen at whitespace
  breaklines=true,                 % sets automatic line breaking
  captionpos=b,                    % sets the caption-position to bottom
  commentstyle=\color{mygreen},    % comment style
  deletekeywords={...},            % if you want to delete keywords from the given language
  escapeinside={\%*}{*)},          % if you want to add LaTeX within your code
  extendedchars=true,              % lets you use non-ASCII characters; for 8-bits encodings only, does not work with UTF-8
  firstnumber=1,                   % start line enumeration with line 1
  numberfirstline=true
  %frame=single,	                   % adds a frame around the code
  keepspaces=true,                 % keeps spaces in text, useful for keeping indentation of code (possibly needs columns=flexible)
  keywordstyle=\color{blue},       % keyword style
  language=Python,                 % the language of the code
  morekeywords={*,...},            % if you want to add more keywords to the set
  numbers=left,                    % where to put the line-numbers; possible values are (none, left, right)
  xleftmargin=2em,
  framexleftmargin=1.5em,
  numbersep=5pt,                   % how far the line-numbers are from the code
  numberstyle=\tiny\color{mygray}, % the style that is used for the line-numbers
  rulecolor=\color{black},         % if not set, the frame-color may be changed on line-breaks within not-black text (e.g. comments (green here))
  showspaces=false,                % show spaces everywhere adding particular underscores; it overrides 'showstringspaces'
  showstringspaces=false,          % underline spaces within strings only
  showtabs=false,                  % show tabs within strings adding particular underscores
  stepnumber=1,                    % the step between two line-numbers. If it's 1, each line will be numbered
  stringstyle=\color{mymauve},     % string literal style
  tabsize=2,	                   % sets default tabsize to 2 spaces
}

% Defining page style
\pagestyle{scrheadings}

\clearscrheadings
\clearscrplain
\clearscrheadfoot

\renewcommand{\headfont}{\normalfont} 
\lohead{\footnotesize \sffamily \doctitle}

\lofoot{%
\begin{minipage}{0.09\textwidth}%
\includegraphics[scale=0.8]{graphics/logo_funding_agency.png}%
\end{minipage}%
\begin{minipage}{0.3\textwidth}%
\footnotesize \sffamily \projectacronym\\\grant%
\end{minipage}}%

\rofoot{\footnotesize \sffamily Page {\bfseries\thepage}~of {\bfseries\pageref{LastPage}}}

% caption settings
\usepackage{caption}
\usepackage[super]{nth}
\numberwithin{figure}{section}
\numberwithin{table}{section}

% hyperref settings
\usepackage[unicode=true]{hyperref}
\usepackage{breakurl}
\hypersetup{breaklinks=true, pdfauthor=\pdfauthor, pdftitle=\pdftitle, colorlinks=true, citecolor=blue, urlcolor=blue, linkcolor=black, pdfborder={0 0 0}}

% don't use monospace font for urls
\urlstyle{same}

% path bibliography so it has a label
\usepackage{etoolbox}
\patchcmd{\thebibliography}{\list}{\label{bib}\list}{}{}

\newcommand\dash{\nobreakdash-\hspace{0pt}}

\usepackage[caption=false,font=normalsize]{subfig}
\captionsetup[subfigure]{position=top,singlelinecheck=off,justification=raggedright}

% Defining style of section titles
\definecolor{sectionTitleBackground}{rgb}{\sectionTitleBackground}
\colorlet{sectionTitleTextColor}{\sectionTitleTextColor}
\definecolor{subsectionTitleBackground}{rgb}{\subsectionTitleBackground}
\colorlet{subsectionTitleTextColor}{\subsectionTitleTextColor}
\colorlet{subsubsectionTitleTextColor}{\subsubsectionTitleTextColor}

\usepackage{tikz}

\setkomafont{section}{\color{sectionTitleTextColor}%
    \bfseries\Large
    \begin{tikzpicture}[overlay]
    \draw[fill=sectionTitleBackground] (-25mm, -5pt) rectangle (\textwidth,16.4pt);
    \end{tikzpicture}}

\setkomafont{subsection}{\color{subsectionTitleTextColor}%
    \bfseries
    \begin{tikzpicture}[overlay]
    \draw[fill=subsectionTitleBackground] (-4pt,-5pt) rectangle (\linewidth,15.4pt);
    \end{tikzpicture}}

\setkomafont{subsubsection}{\color{subsubsectionTitleTextColor}%
    \bfseries
    \begin{tikzpicture}[overlay]
    \draw[fill=subsectionTitleBackground!50!white,draw=white] (-4pt,-5pt) rectangle (\linewidth,15.4pt);
    \end{tikzpicture}}

\newenvironment{myenv}[1]{\mdfsetup{
    frametitle={\color{sectionTitleBackground}\colorbox{white}{\space#1\space}},
    innertopmargin=10pt,
    frametitleaboveskip=-\ht\strutbox,
    frametitlealignment=\center
    }
  \begin{mdframed}}{\end{mdframed}}